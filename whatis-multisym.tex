%+----------------------------------------------------------------------------+
%| SLIDES: 
%| Chapter: Brief introduction to multisymplectic geometry and Homotopy momap
%| Author: Antonio miti
%| Event: Phd Colloquium - What is ... Geometric mechanics?
%+----------------------------------------------------------------------------+

%- HandOut Flag -----------------------------------------------------------------------------------------
\makeatletter
\@ifundefined{ifHandout}{%
  \expandafter\newif\csname ifHandout\endcsname
}{}
\makeatother

%- D0cum3nt ----------------------------------------------------------------------------------------------
\documentclass[beamer,10pt]{standalone}   
%\documentclass[beamer,10pt,handout]{standalone}  \Handouttrue  

\ifHandout
	\setbeameroption{show notes} %print notes   
\fi

	
%- Packages ----------------------------------------------------------------------------------------------
\usepackage{custom-style}


%--Beamer Style-----------------------------------------------------------------------------------------------
\usetheme{toninus}





%---------------------------------------------------------------------------------------------------------------------------------------------------
%- D0cum3nt ----------------------------------------------------------------------------------------------------------------------------------
\begin{document}
%------------------------------------------------------------------------------------------------






%-------------------------------------------------------------------------------------------------------------------------------------------------
\subsection{Multisymplectic geometry}
%-------------------------------------------------------------------------------------------------------------------------------------------------





%-------------------------------------------------------------------------------------------------------------------------------------------------
\begin{frame}[fragile]{Multisymplectic manifolds} %Fragile -->workaround tikzcd
	\begin{defblock}[$n$-plectic manifold ~\emph{(Cantrijn, Ibort, De Le\'on)}]
	\includestandalone[width=0.95\textwidth]{Pictures/Figure_multisym}	
	\end{defblock}
	%
	\begin{defblock}[Non-degenerate $(n+1)$-form]
		\begin{columns}
			\begin{column}{.45\linewidth}
				\centering{
				The $\omega^\flat$ (flat) bundle map is injective.
				}
			\end{column}
			\begin{column}{.5\linewidth}
						\vspace{-.5em}
				\[
				\begin{tikzcd}[column sep= small,row sep=0ex,
				/tikz/column 1/.append style={anchor=base east}]
				    \omega^\flat \colon T M \ar[r]& \Lambda^n T^\ast M \\
  						 (x,u) \ar[r, mapsto]& (x,\iota_{u} \omega_x)						
				\end{tikzcd}	
				\]
			\end{column}
		\end{columns}
	\end{defblock}
	%
	\pause
	\begin{defblock}[Hamiltonian $(n-1)$-forms]
		\begin{displaymath}
			\Omega^{n-1}_{ham}(M,\omega) 	:=
			\biggr\{ \sigma \in  \Omega^{n-1}(M) \; \biggr\vert \; 
				\exists \mathscr{v}_\sigma \in \mathfrak{X}(M) ~:~ 
				\tikz[baseline,remember picture]{\node[rounded corners,
                        fill=orange!5,draw=orange!30,anchor=base]            
            			(target) {$d \sigma = -\iota_{\mathscr{v}_\sigma} \omega$ };
            	}				
				~\biggr\} 
			\end{displaymath}
	\end{defblock}
	%
	%
	\pause
		\tikz[overlay,remember picture]
		{
			\node[rounded corners,
                 fill=orange!5,draw=orange!30,anchor=base]            
            	 (base) at ($(current page.east)-(2.25,1.8)$) [rotate=-0,text width=4cm,align=center] {\footnotesize{\textcolor{red}{Hamilton-DeDonder-Weyl \\equation}}};
		}	
	\begin{tikzpicture}[overlay,remember picture]
    	\path[->] (base.west) edge[bend left,red](target.south west);
    \end{tikzpicture}	
	\pause
	\vfill
	%
	\begin{block}{Examples:}
		\vspace{-.5em}
		\setbeamercovered{transparent}
		\begin{itemize}[<+->]
			\item[$\bullet$] $n=1$ \qquad\qquad\qquad $\Rightarrow$\quad $\omega$ is a symplectic form
			\item[$\bullet$]  $n=(dim(M)-1)$ \quad$\Rightarrow$\quad $\omega$ is a volume form
			%Any oriented $(n+1)$-dimensional manifold is $n$-plectic w.r.t. the volume form.
			%\item[$\bullet$] Let $G$ a semisimple Lie group and $\langle\cdot,\cdot \rangle$  its killing form. Then $\langle [\cdot,\cdot],\cdot \rangle$ extends to a biinvariant multisymplectic form $\omega$.
			\item[$\bullet$] Let $Q$ a smooth manifold, the multicotangent bundle $\Lambda^n T^\ast Q$ is naturally $n$-plectic.%
			\quad
			\textit{(cfr, \href{https://arxiv.org/abs/physics/9801019}{GIMMSY} construction for classical field theories)}
		\end{itemize}
	\end{block}			 
	
%
\end{frame}
\note[itemize]{
	\item Multisymplectic ($n$-plectic) geometry is a generalization of symplectic geometry where a closed, non degenerate $n+1$-form $(n\geq 1)$  takes the place of the symplectic 2-form
	
	\item multisymplectic means \emph{going higher} in the degree of $\omega$
	
	\item non degeneracy means $\iota_v\omega = 0 \Leftrightarrow v=0$.
	
	\item examples 
		\begin{itemize}
			\item[$\bullet$] 1-plectic $=$ symplectic
			\item[$\bullet$] Any oriented $(n+1)$-dimensional manifold is $n$-plectic w.r.t. the volume form.
			\item[$\bullet$] The multicotangent bundle $\Lambda^n T^\ast Q$ is naturally $n$-plectic.
		\end{itemize}
	
	\item We recognize the special class of forms, called Hamiltonian, determining the Hamiltonian vector fields. 
	Not every $n-1$ form admits an Hamiltonian vector field.
	When it exists, non degeneracy guarantees unicity.
	
	\item Observe also that, by degree reason, when $n$ is equal to $1$ or $dim(M)+1$ an injective flat map $\flat$ is also bijective.
	
	\item It is important to stress that mechanical systems are not the only source instances of this class of of structures. 
				e.g. any semisimple Lie groups has associated a 2-plectic structure and any oriented $n+1$ dimensional manifold is naturally $n$-plectic.
				

}
%---------------------------------------------------------------------------------------------------------------------------------------------------








%---------------------------------------------------------------------------------------------------------------------------------------------------
\subsection{Lie $\infty$-algebra of Observables}
\begin{frame}[fragile,t]{Lie $\infty$-algebra of Observables (higher observables) }
	Let be $(M,\omega)$ a $n$-plectic manifold.
	\begin{defblock}[$L_\infty$-algebra of observables ~\emph{(Rogers)}]
		\hspace{.25em} Is a cochain-complex $(L,\{\cdot\}_1)$ \\
		\vspace{-2.5em}
		\begin{center}
		\ifHandout
			\includestandalone{Pictures/Figure_Observables}	
		\else
			\includestandalone{Pictures/Frame_Observables}
		\fi				
		\end{center}
		\onslide<2->{
			\hspace{.25em} with $n$ (skew-symmetric) multibrackets $(2 \leq k \leq n+1)$\\
			\vspace{-1.5em}
			\begin{center}
				\includestandalone{Pictures/Equation_Multibracket}	
			\end{center}
		}
		%
	\end{defblock}
  	\vfill
	\onslide<3->{
		\emph{Higher analogue} of the \emph{Poisson algebra structure} associated to a symplectic mfd.
	\vfill
	\begin{columns}
		\hfill
		\begin{column}{.11\linewidth}	
			If $n>1$:
			
		\end{column}	
		\begin{column}{.8\linewidth}
		\begin{itemize}
			\item[\xmark] \textcolor{red}{we lose} :\quad multiplication of observables, Jacobi equation;
			%\\ \hspace*{4.25em} full-fledged Jacobi equation;
			\item[\cmark] \textcolor{green}{we gain} :\quad brackets with arities different than two,\\
			\hspace*{4.25em}
			 Jacobi equation \emph{up to homotopies}.
		\end{itemize}		
		\end{column}		
	\end{columns}
	}
  \end{frame}
 \note[itemize]{
	\item if symplectic manifolds are the symmetric take on mechanics, Poisson algebras are the algebraic counterpart.
 	\item A Lie algebra is associated to an ordinary symplectic manifold (its Poisson algebra).
	%(Underlying this is a Lie algebra, whose Lie bracket is the Poisson bracket.)
	Similarly, one associates an Lie-$n$ algebra to any $n$-plectic manifold.
 	% https://ncatlab.org/nlab/show/n-plectic+geometry 	 
 	 %https://ncatlab.org/nlab/show/Poisson+bracket+Lie+n-algebra
	 \item Basically, the higher observables algebra is a chunk of the de Rham complex of $M$ with inverted grading( convention employed here) and an extra structure called "multibrackets".
 	\item ( In the 1-plectic case it reduces to the corresponding Poisson algebra of classical observables)
 	\item Rogers associated to any n-plectic mfd a $L\-\infty$ algebra, Zambon generalized it to the pre-n-plectic case.
 	\item Recognize in the definition of $\{\cdot,\ldots,\cdot\}_k$ the contraction with hamiltonian fields $v_\sigma$ w.r.t. $\sigma$.
  	\item Note $	\iota_{v_{\sigma_1}}\cdots\iota_{v_{\sigma_k}} = (-)^{(k-1)+(k-2)+\dots+1}\iota_{v_{\sigma_k}}\cdots\iota_{v_{\sigma_1}} = (-)^{\frac{k(k-1)}{2}}\iota_{v_{\sigma_k}}\cdots\iota_{v_{\sigma_1}}$ 
 	The definition usually find in literature of Rogers multibrackets involves the coefficient $ (-)^{\frac{k(k-1)}{2}} = -\varsigma(k-1) = (-)^{k+1} \varsigma(k)$.
  \item higher observables is Special instance of a more general object  called $L\-\infty$ Algebra...
 }
%------------------------------------------------------------------------------------------------


%-------------------------------------------------------------------------------------------------------------------------------------------------
\subsection{Homotopy comomentum maps}\label{frame:hcmm-main}
\begin{frame}[fragile]{Homotopy comomentum maps}
	Consider a Lie algebra action $v:\mathfrak{g} \to \mathfrak{X}(M)$  \underline{preserving the $n$-plectic form $\omega$}.
	\vfill
	\begin{defblock}[Homotopy comomentum map \emph{(Callies, Fregier, Rogers, Zambon)}]
		\ifHandout
			\includestandalone{Pictures/Figure_Lifting}
		\else
			\includestandalone{Pictures/Frame_Lifting}
		\fi					
	\end{defblock}
	\onslide<4->{
	\begin{lemblock}[HCMM unfolded  \cite{Callies2016}]
			%
			HCMM is a sequence of (graded-skew) multilinear maps:
			\begin{displaymath}
				(f)  = \big\lbrace f_k: \; \Lambda^k{\mathfrak g} \to L^{1-k} \subseteq \Omega^{n-k}(M) 
				~\big\vert~ 0\leq k \leq n+1  \big\rbrace
			\end{displaymath}
			\emph{fulfilling:}%\emph{such that:}
			\begin{itemize}
				\item<5-> $f_0 = 0 $, $f_{n+1} = 0$
				\item<6-> $d f_k (p) = f_{k-1} (
				\tikz[baseline,remember picture]{\node[rounded corners,
                        fill=green!5,draw=green!30,anchor=base]            
            			(target) {$\partial $ };
            	}				
				p)  - (-1)^{\frac{k(k+1)}{2}} \iota(v_p) \omega 
				\qquad\scriptstyle \forall p \in \Lambda^k(\mathfrak{g}),\; \forall k=1,\dots n+1$
			\end{itemize}
		\onslide<7->{
			\tikz[overlay,remember picture]
			{
				\node[rounded corners,
	                 draw=green!30,anchor=base]            
	            	 (base) at ($(current page.east)-(3,3)$) [rotate=-0,align=center] {\footnotesize{\hyperlink{frame:CE-complex}{\emph{Chevalley-Eilenberg boundary op.}}}};
			}	
		\begin{tikzpicture}[overlay,remember picture]
	    	\path[->] (base.west) edge[bend right,green](target.north east);
	    \end{tikzpicture}
	    }
	\end{lemblock}	
	}
	\vfill
\end{frame}
\note[itemize]{
	\item  An infinitesimal symmetry is a lie algebra morphism such that $\mathcal{L}_{v_x} \omega = 0 ~ \forall x \in \mathfrak{g}$.
	\\ (It is also call an infinitesimal multisymplectic action and $v_x$ is the infinitesimal generator of the action, corresponding to $x \in \mathfrak g$.) 
	\item Essentially, admitting a comoment maps mean that $v$ acts via Hamiltonian vector fields.
	\item In mechanical terms, a moment map is a tool associated with a Hamiltonian action of a Lie group on a symplectic manifold, used to construct conserved quantities for the action.(see \ref{frame:HCMMandConserved} in appendix.
}
%-------------------------------------------------------------------------------------------------------------------------------------------------








%----------------------------------------------------------------------------------------------------------------------------------
\end{document}
%----------------------------------------------------------------------------------------------------------------------------------




